\documentclass[letterpaper]{article} % Feel free to change this
\usepackage{graphicx}
\usepackage{caption}
\usepackage{rotating}
\usepackage{tabto}
\usepackage{amsmath}
% comment out the next two lines if it doesn't compile for you
% or you could just install this package as well
\usepackage{vmargin}   
\setmarginsrb{1.0in}{1.0in}{1.0in}{1.0in}{12pt}{10mm}{12pt}{0mm}
\usepackage{fancyhdr}
 
\pagestyle{fancy}
\fancyhf{}
\rhead{CS 590.01 Assignment 1}
\lhead{Alexander Zapata}
\rfoot{Page \thepage}
 
\begin{document}

\title{ECE 538: VLSI System Testing\\
\begin{large}{Assignment 4}\end{large}
\date{April 19, 2019}} % Change this to the date you are submitting
\author{Alexander Zapata}
\maketitle


\vspace{12.0cm}
\section*{Duke Community Standard}

By submitting this \LaTeX{} document, I affirm that
\begin{enumerate}
    \item I have adhered to the Duke Community Standard in completing this assignment.
\end{enumerate}

\newpage 

\section*{Problem 1 {\small Path	Delay	and	Small	Delay	Defect	Testing	using	Synopsys	TetraMax:}}
\subsection*{a. Path Delay Faults}
\subsubsection*{(i)}

\begin{table}[ht]
\centering
\begin{tabular}{|c|c|c|c|c|c|c|}
\hline
Number of Critical Paths & 50      & 100     & 150     & 200     & 250     & 300     \\ \hline
Total Faults             & 50      & 100     & 150     & 200     & 250     & 300     \\ \hline
Detected                 & 44      & 82      & 123     & 128     & 129     & 128     \\ \hline
Test Coverage            & 88.00\% & 82.00\% & 82.00\% & 64.00\% & 51.60\% & 42.67\% \\ \hline
Patterns                 & 9       & 18      & 19      & 19      & 19      & 18      \\ \hline
CPU Time                 & 0.02    & 0.02    & 0.03    & 0.02    & 0.03    & 0.05    \\ \hline
\end{tabular}
\caption{Results for path-delay faults, 0.15ns clock period}
\end{table}


\subsubsection*{(ii)}

\begin{table}[ht]
\centering
\begin{tabular}{|c|c|c|c|c|c|c|}
\hline
Number of Critical Paths & 50      & 100     & 150     & 200     & 250     & 300     \\ \hline
Total Faults             & 50      & 100     & 150     & 200     & 250     & 300     \\ \hline
Detected                 & 44      & 82      & 123     & 128     & 129     & 128     \\ \hline
Test Coverage            & 88.00\% & 82.00\% & 82.00\% & 64.00\% & 51.60\% & 42.67\% \\ \hline
Patterns                 & 9       & 18      & 19      & 18      & 19      & 18      \\ \hline
CPU Time                 & 0.01    & 0.03    & 0.03    & 0.03    & 0.03    & 0.04    \\ \hline
\end{tabular}
\caption{Results for path-delay faults, 0.10ns clock period}
\end{table}
The fault coverage for the 0.15ns/0.10ns path delay fault simulations were exactly the same. Having the same total faults, detected faults, and fault coverage means that--- from one timing to the next--- no additional delay faults were found on the critical paths tested (i.e., the paths not detected in the 0.15ns simulation had significant enough slack to also not be detected in the 0.10ns simulation). Between simulations, there was one more pattern for the 200 critical path simulation with 0.15ns clock than 0.10ns clock. This means that with the faster clock, fewer patterns were necessary to sensitize a delay long enough to detect. The CPU times were roughly the same for each simulation.

\subsection*{b. Small Delay Defects}
\subsubsection*{(i)}

\begin{table}[ht]
\centering
\begin{tabular}{|c|c|c|c|c|c|}
\hline
Slack               & 10\%            & 15\%             & 20\%            & 25\%             & 30\%            \\ \hline
Total Faults        & 4094            & 4094             & 4094            & 4094             & 4094            \\ \hline
Detected            & 3994            & 3994             & 3994            & 3994             & 3994            \\ \hline
Delay Effectiveness & 0.11ns(55.17\%) & 0.165ns(30.75\%) & 0.22ns(50.08\%) & 0.275ns(49.82\%) & 0.33ns(53.68\%) \\ \hline
SDQL                & 6289088.50      & 6126893.50       & 5438607.50      & 4897204.50       & 4477742.50      \\ \hline
CPU Time            & 0.07            & 0.07             & 0.07            & 0.08             & 0.08            \\ \hline
\end{tabular}
\caption{Results for small delay defects, 1.1ns clock period}
\end{table}


\newpage

\subsubsection*{(ii)}

\begin{table}[ht]
\centering
\begin{tabular}{|c|c|c|c|c|c|}
\hline
Slack               & 10\%           & 15\%            & 20\%            & 25\%            & 30\%            \\ \hline
Total Faults        & 4094           & 4094            & 4094            & 4094            & 4094            \\ \hline
Detected            & 3994           & 3994            & 3994            & 3994            & 3994            \\ \hline
Delay Effectiveness & 0.12ns(6.64\%) & 0.18ns(49.23\%) & 0.24ns(25.95\%) & 0.30ns(44.24\%) & 0.36ns(49.65\%) \\ \hline
SDQL                & 5756102.00     & 5206344.50      & 5301291.50      & 4501716.00      & 4122875.25      \\ \hline
CPU Time            & 0.08           & 0.06            & 0.07            & 0.07            & 0.08            \\ \hline
\end{tabular}
\caption{Results for small delay defects, 1.2ns clock period}
\end{table}


\subsubsection*{(iii)}

\begin{table}[ht]
\centering
\begin{tabular}{|c|c|c|c|c|c|}
\hline
Slack               & 10\%            & 15\%            & 20\%            & 25\%            & 30\%            \\ \hline
Total Faults        & 4094            & 4094            & 4094            & 4094            & 4094            \\ \hline
Detected            & 3994            & 3994            & 3994            & 3994            & 3994            \\ \hline
Delay Effectiveness & 0.10ns(44.24\%) & 0.15ns(45.63\%) & 0.20ns(50.85\%) & 0.25ns(51.03\%) & 0.30ns(58.95\%) \\ \hline
SDQL                & 6445672.00      & 5683769.50      & 5567790.00      & 5023122.00      & 4668992.50      \\ \hline
CPU Time            & 0.07            & 0.08            & 0.08            & 0.08            & 0.08            \\ \hline
\end{tabular}
\caption{Results for small delay defects, 1.0ns clock period}
\end{table}


\end{document}
